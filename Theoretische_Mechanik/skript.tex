\documentclass[a4paper, 12pt]{report}

%%%%%%%%%%%%
% Packages %
%%%%%%%%%%%%

\usepackage[german]{babel}
\usepackage[noheader]{../packages/sleek}
\usepackage{../packages/sleek-title}
\usepackage{../packages/sleek-theorems}
\usepackage{../packages/sleek-listings}

\usepackage[textsize=scriptsize]{todonotes}

\usepackage{mathrsfs}

%%%%%%%%%%%%%%
% Title-page %
%%%%%%%%%%%%%%

\logo{../resources/logo/phi-logo.jpg}
\institute{Universität Bayreuth}
\faculty{Physik}
%\department{Department of Anything but Psychology}
\title{Theoretische Mechanik}
\subtitle{Stoffsammlung}
\author{von\\Moritz Schramm}%\textit{Author}\\François \textsc{Rozet}}
%\supervisor{Linus \textsc{Torvalds}}
%\context{Well, I was bored...}
\date{Sommersemester 2021\\ \small basierend auf dem Lehrbuch von Fließbach und dem Vorlesungsskript von Prof. Gekle}

%%%%%%%%%%%%%%%%
% Bibliography %
%%%%%%%%%%%%%%%%

\addbibresource{../resources/bib/references.bib}

%%%%%%%%%%
% Others %
%%%%%%%%%%

\lstdefinestyle{latex}{
    language=TeX,
    style=default,
    %%%%%
    commentstyle=\ForestGreen,
    keywordstyle=\TrueBlue,
    stringstyle=\VeronicaPurple,
    emphstyle=\TrueBlue,
    %%%%%
    emph={LaTeX, usepackage, textit, textbf, textsc}
}

\FrameTBStyle{latex}

\def\tbs{\textbackslash}

\newcommand{\dd}{\mathrm{d}} % derivative
\newcommand{\R}{\mathbb{R}} % real numbers
\newcommand{\N}{\mathbb{N}}
\newcommand{\Z}{\mathbb{Z}}
\newcommand{\Q}{\mathbb{Q}}
\newcommand{\C}{\mathbb{C}}
\newcommand{\st}{{:}\ }	% a ":" (such that) with correct spacing

\renewcommand{\labelitemi}{\tiny$\square$}

%%%%%%%%%%%%
% Document %
%%%%%%%%%%%%

\begin{document}
    \maketitle
    \romantableofcontents

    \chapter{Grundlagen}
    test
    
    \chapter{Lagrange Formalismus}
    
\section{Differentiation}

\begin{framedprop}[Differentiation mit Vektoren und Differentialen]
	\begin{enumerate}
		\item[(i)] $\frac{\dd}{\dd t} (\mathbf{a} + \mathbf{b}) = \frac{\dd}{\dd t} \mathbf{a} + \frac{\dd}{\dd t} \mathbf{b}$
		
		\item[(ii)] $\frac{\dd}{\dd t} (\mathbf{a} \cdot \mathbf{b}) = \frac{\dd}{\dd t} \mathbf{a} \cdot \mathbf{b} + \mathbf{a} \cdot \frac{\dd}{\dd t} \mathbf{b}$
		
		\item[(iii)] $\frac{\dd}{\dd t} (\mathbf{a} \times \mathbf{b}) = \frac{\dd}{\dd t} \mathbf{a} \times \mathbf{b} + \mathbf{a} \times \frac{\dd}{\dd t} \mathbf{b}$
		
		\item[(iv)] $\frac{\dd}{\dd t} (f \mathbf{b}) = \frac{\dd}{\dd t} f \  \mathbf{a} + f \  \frac{\dd}{\dd t} \mathbf{a}$
	\end{enumerate}

Alle Regeln gelten auch für Differentiale, z.B. $\dd (\mathbf{a} \cdot \mathbf{b}) = \dd \mathbf{a} \cdot \mathbf{b} + \mathbf{a} \cdot \dd \mathbf{b}$
\end{framedprop}

\section{Krummlinige Bewegung}

\begin{framedprop}[Grundlegende Bewegungsvektoren]
Tangenteneinheitsvektor $\mathbf{t}$, Normaleneinheitsvektor $\mathbf{n}$ und Binormale $\mathbf{b}$:
\[
\mathbf{t} = \frac{\dd \mathbf{r}}{\dd s} = \left|\frac{\dd \mathbf{r}}{\dd t}\right|^{-1} \frac{\dd \mathbf{r}}{\dd t}  \qquad
\mathbf{n} = \left|\frac{\dd \mathbf{t}}{\dd s}\right|^{-1} \frac{\dd \mathbf{t}}{\dd s}  \qquad
\mathbf{b} = \mathbf{t} \times \mathbf{n}
\]
Krümmung $\kappa$ und Krümmungsradius $\rho$:
\[
\kappa = \left|\frac{\dd \mathbf{t}}{\dd s}\right| \qquad \rho = \frac{1}{\kappa}
\]
Nützliche Zusammenhänge:
\[
\dd\mathbf{r} = \frac{\dd \mathbf{r}}{\dd t}\dd t \qquad \dd s = |\dd\mathbf{r}| \qquad s = \int_{0}^{s} \dd s' = \int_{0}^{t} \left| \frac{\dd \mathbf{r}}{\dd t} \right| \dd t
\]

\end{framedprop}

\section{Nabla Kalkül}

\begin{frameddefn}[Nabla und Laplace Operator]
\[
\nabla = \left( \frac{\partial}{\partial x},\, \frac{\partial}{\partial y},\, \frac{\partial}{\partial z} \right)^T
\]
\[
\Delta = \nabla^2 = \frac{\partial^2}{\partial x^2} + \frac{\partial^2}{\partial y^2} + \frac{\partial^2}{\partial z^2}
\]
\end{frameddefn}

\begin{frameddefn}[Hesse Matrix]
\[
(\nabla\nabla f)_{ij} = 
\left(\begin{array}{rrr} 
	\frac{\partial^2 f}{\partial x^2} & \frac{\partial^2 f}{\partial x \partial y} & \frac{\partial^2 f}{\partial x \partial z} \\ 
	\frac{\partial^2 f}{\partial y \partial x} & \frac{\partial^2 f}{\partial y^2} & \frac{\partial^2 f}{\partial y \partial z} \\ 
	\frac{\partial^2 f}{\partial z \partial x} & \frac{\partial^2 f}{\partial z \partial y} & \frac{\partial^2 f}{\partial z^2} \\ 
\end{array}\right)
\]
\end{frameddefn}

\begin{frameddefn}[Gradient, Divergenz und Rotation]
\[
\textrm{grad}\, f(\mathbf{r}) = \nabla f(\mathbf{r}) \qquad
\textrm{div}\, \mathbf{A}(\mathbf{r}) = \nabla \cdot \mathbf{A}(\mathbf{r}) \qquad
\textrm{rot}\, \mathbf{A}(\mathbf{r}) = \nabla \times \mathbf{A}(\mathbf{r})
\]
Der Gradient gibt die Richtung des steilsten Anstiegs von $f$ an, der Betrag des Gradienten gibt an, wie groß diese Steigung ist.\\Die Divergenz ist ein Skalarfeld und gibt an, wie sehr die Vektoren an einem Punkt auseinanderstreben. Bei einem Stömungsfeld ist die Divergenz die Quelldichte. Senken haben negative Divergenz. Bei einer Divergenz gleich null ist das Feld quellfrei.\\Die Rotation gibt die lokale Wirbelstärke an. Ist diese gleich null, so ist das Feld wirbelfrei.
\end{frameddefn}


\section{Krummlinige Koordinaten}

Die Punkte im $\R^3$ seien anstatt durch die kartesischen Koordinaten $x,y,z$ durch die neuen Koordinaten $u,v,w$ gekennzeichnet.

\begin{frameddefn}[Jacobi Matrix]
\[
\mathbf{J} =
\left(\begin{array}{rrr} 
	\frac{\partial x}{\partial u} & \frac{\partial x}{\partial v} & \frac{\partial x}{\partial w} \\ 
	\frac{\partial y}{\partial u} & \frac{\partial y}{\partial v} & \frac{\partial y}{\partial w} \\ 
	\frac{\partial z}{\partial u} & \frac{\partial z}{\partial v} & \frac{\partial z}{\partial w} \\ 
\end{array}\right)
\textrm{ mit Inverse: }
\mathbf{J}^{-1} = 
\left(\begin{array}{rrr} 
	\frac{\partial u}{\partial x} & \frac{\partial u}{\partial y} & \frac{\partial u}{\partial z} \\ 
	\frac{\partial v}{\partial x} & \frac{\partial v}{\partial y} & \frac{\partial v}{\partial z} \\ 
	\frac{\partial w}{\partial x} & \frac{\partial w}{\partial y} & \frac{\partial w}{\partial z} \\ 
\end{array}\right)
\]
\end{frameddefn}

\begin{frameddefn}[Funktionaldeterminante]
\[	
\frac{\partial (x,y,z)}{\partial (u,v,w)} = \det(\mathbf{J})
\]
\end{frameddefn}

\begin{framedprop}[Einheitsvektoren von krummlinigen Koordinaten]
\[
\mathbf{e}_u = \left| \frac{\partial \mathbf{r}}{\partial u} \right|^{-1} \frac{\partial \mathbf{r}}{\partial u} \qquad
 \mathbf{e}_v = \left| \frac{\partial \mathbf{r}}{\partial v} \right|^{-1} \frac{\partial \mathbf{r}}{\partial v} \qquad
 \mathbf{e}_w = \left| \frac{\partial \mathbf{r}}{\partial w} \right|^{-1} \frac{\partial \mathbf{r}}{\partial w}
\]
Im Allgemeinen sind die Einheitsvektoren vom Ort $\mathbf{r}$ abhängig. Nur in Spezialfällen (z.B. bei kartesischen Koordinaten) sind diese unabhängig. Die drei Einheitsvekoren bilden ein lokales Dreibein. Oft praktisch sind krummlinig-orthogonal Koordinaten, d.h. $\mathbf{e}_{u_{i}} \cdot \mathbf{e}_{u_{j}} = \delta_{ij}$.
\end{framedprop}

\begin{framedprop}[Wichtige krummlinige Koordinaten]
	\begin{itemize}
		\item Polarkoordinaten
		\item Zylinderkoordinaten
		\item Kugelkoordinaten
	\end{itemize}
\end{framedprop}

\section{Kurvenintegrale}

\begin{framedprop}[Kurvenintegral durch Substitution lösen]
	Inhalt...
\end{framedprop}

\begin{framedthm}[Unabhängigkeit vom Weg]

\end{framedthm}

\section{Volumen- und ebene Flächenintegrale}

\section{Flächenintegrale}

\begin{framedprop}[Parametrisierung]
	Inhalt...
\end{framedprop}

\section{Integralsätze}

\begin{framedthm}[Satz von Gauß (Divergence Theorem)]
	Sei $\mathbf{A}(\mathbf{r})$ ein Vektorfeld, $V$ ein Volumen mit geschlossener Oberfläche $\partial V$ (Normalen nach außen gerichtet). Dann gilt:
	\[
	\oint_{\partial V} \dd \mathbf{f} \cdot \mathbf{A}(\mathbf{r}) = \int_{V} \dd \mathbf{r} \,\, \nabla \cdot \mathbf{A}(\mathbf{r}) 
	\] 
\end{framedthm}

\begin{framedthm}[Satz von Stokes]
	Sei $\mathbf{A}(\mathbf{r})$ ein stetig differenzierbares Vektorfeld, $\partial F$ eine glatte Kurve und $F$ eine über $\partial F$ gespannte Fläche. Dann gilt:
	\[
	\oint_{\partial F} \dd \mathbf{r} \cdot \mathbf{A}(\mathbf{r}) = \int_F \dd \mathbf{f} \cdot (\nabla \times \mathbf{A}(\mathbf{r}))
	\]
\end{framedthm}


    
	\chapter{Variationskalkül}
	\begin{frameddefn}[Reihendefinition]
	Sei $(z_n)_{n\geq 1}$ eine Folge komplexer Zahlen. Dann heißt $s_m = \sum_{k=1}^{m} z_k$ Partialsumme von $(z_n)_{n\geq 1}$. Die Folge $(s_m)_{m\geq 1}$ heißt \textit{Reihe} mit den Gliedern $(z_n)_{n\geq 1}$ und wird mit $\sum_{k=1}^{\infty} z_k$ bezeichnet.
\end{frameddefn}

\begin{framedthm}[Geometrische Reihe]
	Sei $q \in \C$ mit $|q| < 1$. Dann konvergiert die sogenannte \textit{geometrische Reihe} $\sum_{k=0}^{\infty} q^k = \frac{1}{1-q} \in \C$ absolut.
\end{framedthm}

\begin{framedthm}
	Eine notwendige, aber nicht hinreichende Bedingung für die Konvergenz einer Reihe $\sum_{n=1}^{\infty} a_n$ ist, dass $a_n$ eine Nullfolge ist. Es gilt also:
	\begin{align*}
		\lim_{n \to \infty} a_n \neq 0 \ \Rightarrow\ \sum_{n=1}^{\infty} a_n \textrm{ divergiert}
	\end{align*}
\end{framedthm}

\section{Konvergenz Kriterien für Reihen}

\begin{framedthm}[Cauchysches Konvergenzkriterium für Reihen]
	\begin{align*}
		\sum_{n=1}^{\infty} z_n \textrm{ konvergiert} \ \iff\  \forall \varepsilon > 0\st\exists n_0 \in \N\st \forall n,m \geq n_0\st \biggl|\sum_{k=n}^{m} z_k \biggr| < \varepsilon
	\end{align*}
\end{framedthm}

\begin{frameddefn}[Absolute Konvergenz]
	Die Reihe $\sum_{k=0}^{\infty} z_n$ ist \textit{absolut konvergent}, wenn die Reihe $\sum_{k=0}^{\infty} |z_k|$ konvergiert. Jede absolute konvergente Reihe konvergiert.
\end{frameddefn}

\begin{framedthm}[Majoranten- und Minorantenkriterium] \label{majok}
	Seien $\sum_{n=0}^{\infty} a_n$ und $\sum_{n=0}^{\infty} b_n$ zwei reelle Reihen.
	\begin{enumerate}
		\item[(i)] Majorantenkriterium:\\ Falls $\sum_{n=0}^{\infty} b_n$ konvergiert und $\exists N \in \N\st\forall n \geq N\st |a_n| \leq b_n$, dann konvergiert auch $\sum_{n=0}^{\infty} a_n$ absolut.
		\item[(i)] Minorantenkriterium:\\ Wenn $\forall k \in \N\st a_k \geq 0$, $\sum_{n=0}^{\infty} b_n$ divergiert und $\exists N \in \N\st\forall n \geq N\st a_n \geq b_n \geq 0$, dann divergiert auch $\sum_{n=0}^{\infty} a_n$.
	\end{enumerate}
\end{framedthm}

\begin{framedthm}[Leibnizsches Konvergenzkriterium]
	Sei $(a_n)_{n\geq 1}$ eine monoton fallende reelle Folge, sodass $\forall n \geq 1\st a_n \geq 0$ und\\ $\lim\limits_{n \to \infty} a_n = 0$. Dann konvergiert die sogenannte \textit{alternierende Reihe} $\sum_{n=1}^{\infty} (-1)^n a_n$.
\end{framedthm}

\begin{framedthm}
	Sei $(a_n)_{n\geq 1}$ eine monoton fallende reelle Folge, sodass $\forall n \geq 1\st a_n \geq 0$. Dann konvergiert die Reihe $\sum_{n=1}^{\infty} a_n \ \iff \ \sum_{k=0}^{\infty} 2^k a_{2^k}$ konvergiert.
\end{framedthm}

\section{Reihen mit komplexen Gliedern}

\begin{framedthm}[Majoranten- und Minorantenkriterium für komplexe Reihen]
	Das Majoranten- bzw. Minorantenkriterium (Satz \ref{majok}) gilt auch, wenn $(a_n)_{n\geq 1}$ eine komplexe Folge ist.
\end{framedthm}

\begin{framedthm}[Cauchyscher Test]
	Sei $\sum a_n$ eine komplexe Reihe mit $\alpha = \lim\limits_{n \to \infty} \sup |a_n|^{\frac{1}{n}}$. Dann gilt:
	\begin{enumerate}
		\item $\alpha < 1$: $\sum a_n$ konvergiert absolut
		\item $\alpha > 1$: $\sum a_n$ divergiert
		\item $\alpha = 1$: keine Aussage möglich
	\end{enumerate}
\end{framedthm}

\begin{framedthm}[d'Alembertsches Quotientenkriterium]
	Sei $\sum z_n$ eine komplexe Reihe mit $\alpha = \lim\limits_{n \to \infty} \bigl| \frac{a_{n+1}}{a_n}\bigr|$. Dann gilt:
	\begin{enumerate}
		\item $\alpha < 1$: $\sum z_n$ konvergiert absolut
		\item $\alpha > 1$: $\sum z_n$ divergiert
		\item $\alpha = 1$: keine Aussage möglich
	\end{enumerate}
\end{framedthm}

\begin{frameddefn}[Potenzreihen]
	Seien $z_0, z \in \C$ und $(c_n)_{n\geq 0}$ eine Folge komplexer Zahlen. Reihen der Gestalt\\ $\sum_{n=0}^{\infty} c_n (z-z_0)^n$ werden Potenzreihen genannt.
\end{frameddefn}

\begin{framedthm}[Cauchy-Hadamard: Konvergenz von Potenzreihen]
	Sei $\sum_{n=0}^{\infty} c_n (z-z_0)^n$ eine Potenzreihe. Dann gilt:
	\begin{enumerate}
		\item[(i)] Die Potenzreihe konvergiert innerhalb des Kreises:
		\begin{align*}
		|z-z_0| < \frac{1}{\lim\limits_{n \to \infty} \sup |c_n|^\frac{1}{n}}
		\end{align*}
		\item [(ii)] Sie divergiert außerhalb des Kreises
		\item[(iii)] Auf dem Kreisrand ist keine Aussage möglich
	\end{enumerate}
\end{framedthm}

\section{Umgeordnete Reihen}

\begin{frameddefn}[$\tau$-umgeordnete Reihe]
	Sei $\tau\st \N \to \N$ eine bijektive Abbildung. Dann ist $\sum_{n=1}^{\infty} z_{\tau(n)}$ die $\tau$-umgeordnete Reihe.
\end{frameddefn}

\begin{framedthm}[Umordnungssatz]
	Sei $\sum_{n=1}^{\infty} z_n$ eine komplexe, absolut konvergente Reihe. Dann konvergiert auch jede Umordnung dieser Reihe gegen denselben Grenzwert.
\end{framedthm}

\begin{framedthm}[Riemannscher Umordnungssatz]
	Sei $\sum_{n=1}^{\infty} a_n$ eine konvergente, aber \textbf{nicht} absolut konvergente Reihe reeller Zahlen. Dann gilt:
	\begin{enumerate}
		\item[(i)] Sei $c \in \R$ beliebig. Dann $\exists \tau\st \N \to \N$ bijektive Abbildung, sodass $\sum_{n=1}^{\infty} a_{\tau(n)} = c$
		\item[(ii)] $\exists \tau_{+}\st \N \to \N, \tau_{-}\st \N \to \N$ bijektive Abbildungen, sodass $\sum_{n=1}^{\infty} a_{\tau_{+}(n)} = +\infty$ und $\sum_{n=1}^{\infty} a_{\tau_{-}(n)} = -\infty$
	\end{enumerate}
\end{framedthm}

\begin{framedthm}[Cauchy-Produkt von Reihen]
	Seien $\sum_{n=0}^{\infty} z_n$ und $\sum_{n=0}^{\infty} w_n$ absolut konvergente Reihen. Für $n\geq 0$ mit wird definiert: \[c_n = \sum_{k=0}^{n} z_{n-k} w_k\]
	Dann ist \[\sum_{n=0}^{\infty} c_n = \left(\sum_{n=0}^{\infty} z_n\right)\left(\sum_{n=0}^{\infty} w_n\right)\] absolut konvergent.
\end{framedthm}

\begin{framedthm}[Eulersche Zahl]
	Es gilt:
	\[\lim_{n \to \infty} \left(1+ \frac{1}{n}\right)^n = e\]
	sowie $e \in \R \setminus \Q$.
\end{framedthm}

\begin{framedthm}[Eigenschaften der Exponentialfunktion]
	Es gilt:
	\[\exp(z) = \sum_{n=0}^{\infty} \frac{z^n}{n!}\]
	sowie:
	\[\exp(z_1+z_2) = \exp(z_1)\cdot\exp(z_2)\]
\end{framedthm}

\begin{framedthm}[Eulersche Formel]
	\[\forall z \in \C\st \exp(iz) = \cos(z) + i \sin(z)\]
\end{framedthm}

	
	\chapter{Zentralpotenzial}
	\section{Bezeichnungen und Definitionen}

Im Folgenden wird mit $\K$ ($\R$ oder $\C$) der zu betrachtente Körper bezeichnet. Sei $M \subset \K$ der Definitionsbereich. Eine Funktion $f\st M \to \K$ ordnet jedem Element $x \in M$ ein Element $y \in \K$ zu ($f(x) = y$).\\
Die Menge $f(M) = \{y\in \K \ |\ \exists x \in M\st f(x)=y\}$ wird als Wertemenge order auch Bild der Funktion f bezeichnet. Allgemeiner heißt $f(A) = \{y\in \K \ |\ \exists a \in A\st f(a)=y\}$, wenn $A \subset M$, das Bild von A unter f.\\
Dabei bezeichnet $f_{|A}\st A \to \K$ mit $f_{|A}(a) = f(a)$ die Restriktion von f. 

\begin{frameddefn}[Operationen mit Funktionen]
	Seien $f,g\st M \to \K$.
	\begin{enumerate}
		\item[(i)] $(f+g)(x) = f(x) + g(x)$
		\item[(ii)] $(f\cdot g)(x) = f(x) \cdot g(x)$
		\item[(iii)] $(\lambda \cdot f)(x) = \lambda\cdot f(x)$ wobei $\lambda \in \K$
		\item[(iv)] $\frac{f}{g}\st M\setminus\{z \in M\ |\ g(z)=0\} \to \K$ mit $\left(\frac{f}{g}\right)(x)= \frac{f(x)}{g(x)}$
	\end{enumerate}
\end{frameddefn}

\begin{frameddefn}[Komposition von Funktionen]
	Sei $f\st M \to \K$ und $g\st E \to \K$ sodass $f(M) \subset E$. Dann bezeichnet $(g \circ f)\st M \to \K$ mit $(g \circ f)(x)=g(f(x))$ die Komposition der Funktionen $f$ und $g$.
\end{frameddefn}

\newpage
\section{Stetigkeit von Funktionen}

\begin{frameddefn}[Stetigkeit einer Funktion]
	Sei $D \subset \K$ und $f\st D \to \K$ sowie $z_0 \in D$.
	\begin{align*}
		f \textrm{ stetig in } z_0 \ \iff \ \forall \varepsilon > 0\st \exists \delta > 0\st \forall z \in D\st |z-z_0| < \delta \ \Rightarrow\ |f(z)-f(z_0)| < \varepsilon
	\end{align*}
	$f$ heißt stetig, wenn $f$ in jedem $z_0 \in D$ stetig ist. 
\end{frameddefn}

\begin{framedthm}[Folgenkriterium]
	Sei $f\st D \to \K$ und $z_0 \in D$.
	\begin{align*}
		f \textrm{ stetig in } z_0 \ \iff \ \forall (z_n)_{n\geq 1} \subset D\st \lim_{n \to \infty} z_n = z_0 \ \Rightarrow\ \lim_{n \to \infty} f(z_n) = f(z_0)
	\end{align*}
\end{framedthm}

\begin{framedthm}[Rationale Operationen auf stetigen Funktionen]
	Seien $f,g\st D \to \K$ mit $D \subset \K$ beliebig. Wenn $f,g$ in $z_0 \in D$ stetig sind, dann gilt:
	\begin{enumerate}
		\item[(i)] $f+g$ ist stetig in $z_0$
		\item[(ii)] $f\cdot g$ ist stetig in $z_0$
		\item[(iii)] $\lambda\cdot f$ ist stetig in $z_0$
		\item[(iv)] falls $g(z_0) \neq 0$ gilt $\frac{f}{g}$ ist stetig in $z_0$
	\end{enumerate}
\end{framedthm}

\begin{framedthm}[Komposition von stetigen Funktionen]
	Sei $f\st D \to \K$ und $g\st E \to \K$ sodass $f(D) \subset E$. Sei außerdem $f$ in $z_0 \in D$ stetig, sowie $g$ in $w_0 = f(z_0) \in E$ stetig. Dann ist die Funktion $g \circ f$ in $z_0$ stetig.
\end{framedthm}

\section{Grenzwerte von Funktionen}

\begin{frameddefn}[Häufungspunkte einer Menge]
	Sei $M \subset \K$ eine beliebige Menge. Ein Punkt $p \in \K$ ist ein Häufungspunkt der Menge M, wenn $\forall \varepsilon > 0$ die Menge $\{z \in \K\ |\ |z-p| < \varepsilon\}$ eine unendliche Teilmenge von M enthält.
\end{frameddefn}

\begin{frameddefn}[Grenzwert einer Funktion]
	Sei $f\st D \to \K$ und $z_0 \in \K$ ein Häufungspunkt von $D$. Dann ist $A \in \K$ der Grenzwert von $f$, wenn $z$ gegen $z_0$ strebt, falls gilt:
	\begin{align*}
		\forall \varepsilon > 0\st \exists \delta > 0\st \forall z \in D\st 0 < |z-z_0| < \delta \ \Rightarrow\ |f(z) - A| < \varepsilon
	\end{align*}
	\begin{align*}
		\lim_{z \to z_0} f(z) = A
	\end{align*}
\end{frameddefn}


\begin{framedthm}
	$\lim\limits_{z \to z_0} f(z) = A$ gilt genau dann, wenn für jede Folge $(z_n)_{n\geq 1}$ von Elementen aus\\ $D\setminus\{z_0\}$ die gegen $z_0$ konvergieren, die Folge $(f(z_n))_{n\geq 1}$ gegen $A$ konvergiert.
\end{framedthm}

\begin{frameddefn}[Beschränkheit einer Funktion]
	Sei $f\st D \to \K$.
	\begin{align*}
		f \textrm{ ist beschränkt} \ \iff\ \exists M > 0\st \forall z \in D\st |f(z)| \leq M
	\end{align*}
	Ist $f$ beschränkt, wird
	\begin{align*}
		||f||_{C^o} = \sup\{|f(z)|\ \big|\ z \in D)\}
	\end{align*}
	als Supremumsnorm von f bezeichnet.
\end{frameddefn}

\begin{framedthm}[Eigenschaften der Supremumsnorm]
	Seien $f,g$ beschränkte Funktionen. Dann gilt:
	\begin{enumerate}
		\item[(i)] $||f||_{C^o} = 0 \ \iff\ f = 0$
		\item[(ii)] $||\lambda\cdot f||_{C^o} = |\lambda|||f||_{C^o}$
		\item[(iii)] $||f+g||_{C^o} \leq ||f||_{C^o} + ||g||_{C^o}$
	\end{enumerate}
\end{framedthm}

\begin{frameddefn}[Konvergent normale Reihen]
	Eine Reihe $\sum_{n=1}^{\infty} f_n$ von Funktionen $f_n\st D \to \K$ heißt \textit{konvergent normal} genau dann, wenn $\forall n \geq 1\st f_n$ beschränkt ist und $\sum_{n=1}^{\infty}||f_n||_{C^o}$ konvergent.
\end{frameddefn}

\begin{framedthm}
	Sei $\sum_{n=0}^{\infty}f_n$ konvergent normal. Für $z \in D$ setze $f(z) = \sum_{n=0}^{\infty}f_n(z)$. Dann ist die Funktion $f\st D \to \K$ stetig.
\end{framedthm}

\begin{framedquest}[Potenzreihen sind konvergent normal]
	Sei $\sum_{n=0}^{\infty} a_n (z-z_0)^n$ eine Potenzreihe mit Konvergenzradius $|z-z_0| < R$. Dann ist die Potenzreihe im Konvergenzradius konvergent normal.
\end{framedquest}

\section{Globale Eigenschaften stetiger Funktionen}

\begin{framedthm}[Zwischenwertsatz von Bolzano-Cauchy]
	Sei $f\st[a,b] \to \R$ eine stetige Funktion mit $f(a)\cdot f(b) < 0$. Dann $\exists p \in\  ]a,b[\st f(p)=0$.
\end{framedthm}


\begin{framedthm}
	Sei $f:[a,b] \to \R$ stetig. Dann ist $f$ beschränkt und\\
	$\exists x_m, x_M \in [a,b]\st f(x_m) = \inf\{f(x)\ |\ x \in [a,b]\}$ sowie $f(x_M) = \sup\{f(x)\ |\ x \in [a,b]\}$.
\end{framedthm}


\begin{frameddefn}[Gleichmäßig stetig]
	Eine Funktion $f: I \to \R$ heißt in I \textit{gleichmäßig stetig} wenn gilt:
	\begin{align*}
		\forall \varepsilon > 0\st \exists \delta > 0\st \forall x_1, x_2 \in I: |x_1 - x_2| < \delta \ \Rightarrow\ |f(x_1) - f(x_2)| < \varepsilon
	\end{align*}
\end{frameddefn}


\begin{framedthm}
	Jede auf einem Intervall $[a,b]$ mit $a,b \in \R$ stetige Funktion $f\st [a,b] \to \R$ ist gleichmäßig stetig.
\end{framedthm}


\begin{frameddefn}
	Sei $f\st]a,\infty[\to\R$, dann gilt:
	\begin{enumerate}
		\item[(i)] $\lim\limits_{x \to \infty} f(x) = \alpha \in \R \ \iff\ \forall \varepsilon>0\st \exists N \in \R\st \forall x > \max(a,N)\st |f(x)-\alpha| < \varepsilon$
		\item[(ii)] $\lim\limits_{x \to \infty} f(x) = \infty \ \iff\ \forall M>0\st \exists K \in \R\st \forall x > K\st f(x) > M$
		\item[(iii)] $\lim\limits_{x \to \infty} f(x) = -\infty \ \iff\ \forall M>0\st \exists K \in \R\st \forall x > K\st f(x) < M$
	\end{enumerate}
	Für $\lim\limits_{x \to -\infty} f(x)$ ähnlich.
\end{frameddefn}


\begin{frameddefn}[Monotone Funktionen]
	Sei $M \subset \R$ eine Menge, $f\st M \to \R$ eine Funktion. Dann heißt f:
	\begin{itemize}
		\item \textit{monoton wachsend} wenn $\forall x_1, x_2 \in M\st x_1 < x_2 \Rightarrow f(x_1) \leq f(x_2)$
		\item \textit{streng monoton wachsend} wenn $\forall x_1, x_2 \in M\st x_1 < x_2 \Rightarrow f(x_1) < f(x_2)$
		\item \textit{monoton fallend} wenn $\forall x_1, x_2 \in M\st x_1 < x_2 \Rightarrow f(x_1) \geq f(x_2)$
		\item \textit{streng monoton fallend} wenn $\forall x_1, x_2 \in M\st x_1 < x_2 \Rightarrow f(x_1) > f(x_2)$
	\end{itemize}
\end{frameddefn}

\begin{framedthm}
	Sei $f\st [a,b] \to \R$ eine stetige Funktion. Dann ist $f$ genau dann injektiv wenn $f$ streng monoton ist.
\end{framedthm}

\begin{frameddefn}[Umkehrabbildung]
	Seien $M_1, M_2 \subset \R$ und $f\st M_1 \to M_2$ bijektiv. Dann ist $g\st M_2 \to M_1$ genau dann die Umkehrabbildung (Inverse, $g=f^{-1}$), wenn $\forall y \in M_2\st (f \circ g)(y) = y$ und\\ $\forall x \in M_1\st (g \circ f)(x) = x$.
\end{frameddefn}

\begin{framedthm}
	Sei $f\st [a,b] \to \R$ eine stetige, streng monotone Funktion. Dann ist $f([a,b]) = J \subset \R$ bijektiv und $f^{-1}\st J \to [a,b]$ ist auch stetig und monoton.
\end{framedthm}

\section{Landau Symbole}

\begin{frameddefn}[Klein \textit{o} und groß $\mathcal{O}$]
	Sei $f,g \st ]a,\infty[ \to \R$.
	\begin{itemize}
		\item $f(x) = \textit{o}(g(x))$ für $x \to \infty$ wenn\\ $\forall \varepsilon > 0 \st \exists R > 0\st \forall x > \max(R,a)\st |f(x)| < \varepsilon |g(x)|$
		\item $f(x) = \mathcal{O}(g(x))$ für $x \to \infty$ wenn\\ $\exists c > 0 \st \exists R \in \R \st \forall x > R\st |f(x)| \leq c |g(x)|$
	\end{itemize}
	Sei $f,g \st I \to \R$ mit $I \subset \R$. Dann ist:
	\begin{itemize}
		\item $f(x) = \textit{o}(g(x)))$ für $x \to x_0$ wenn \\
		$\forall \varepsilon > 0 \st \exists \delta > 0 \st \forall x \in I \ \cap\  ]x_0 - \delta, x_0 + \delta[\st |f(x)| < \varepsilon |g(x)|$
		\item $f(x) = \mathcal{O}(g(x)))$ für $x \to x_0$ wenn \\
		$\exists c > 0 \st \exists \delta > 0 \st \forall x \in I \ \cap\  ]x_0 - \delta, x_0 + \delta[\st |f(x)| < c |g(x)|$
	\end{itemize}
\end{frameddefn}

\section{Logarithmus}

\begin{framedthm}
	Die Exponentialfunktion $\exp \st \R \to \R$ ist stetig, streng monoton wachsend und\\ $\exp(\R) =\  ]0,\infty[$.\\Die Umkehrfunktion heißt (natürlicher) Logarithmus $\log\st ]0,\infty[\, \to \R$.
\end{framedthm}

\begin{framedthm}[Eigenschaften des Logarithmus]
	\begin{itemize}
		\item $\forall x,y \in\  ]0,\infty[\st \log(xy) = \log(x) + \log(y)$
		\item $\log(1) = 0$
		\item $\log(x) > 0 \ \iff\ x > 1$
	\end{itemize}
\end{framedthm}

\begin{frameddefn}[Potenzen einer positiv reellen Zahl]
	Sei $a \in\  ]0,\infty[$, $z \in \C$. Dann ist $a^z = \exp(z \log(a))$.
\end{frameddefn}

\begin{framedthm}
	Die Funktion $f(x) = a^x$, $f\st \R \to\, ]0,\infty[$ ist stetig und:
	\begin{itemize}
		\item $\forall x,y \in \R\st a^{x+y} = a^x a^y$
		\item $\forall n \in \N\st a^{\frac{1}{n}} = \sqrt[n]{a}$
		\item $\forall x,y \in \R\st (a^x)^y = a^{xy}$
	\end{itemize}
\end{framedthm}

\begin{framedthm}
	Sei $f\st \R \to \R$ eine stetige reelle Funktion mit $\forall x,y \in \R\st f(x+y) = f(x) f(y)$.
	Dann ist f entweder $f(x) = a^x$ mit $a \in \R_+$ oder $\forall x \in \R\st f(x) = 0$.
\end{framedthm}

\begin{framedquest}[Grenzwerte der Logarithmus Funktion]
	\begin{enumerate}
		\item [(i)] $\lim\limits_{x \to 0} \log x = - \infty$ wobei $x > 0$
		\item [(ii)] $\lim\limits_{x \to \infty} \log x = \infty$
		\item [(iii)] Sei $\alpha \in \R_+$ dann $\lim\limits_{x \to 0} x^{\alpha} = 0$ wobei $x > 0$
		\item [(iv)] $\lim\limits_{x \to \infty} \frac{\log x}{x^{\alpha}} = 0$
	\end{enumerate}
\end{framedquest}

\newpage
\section{Trigonometrische Funktionen}

\begin{frameddefn}[Sinus und Kosinus]
	Da für $z \in \C$ $\exp(iz) = \cos(z) + i \sin(z)$ gilt:
	\[
	\cos(z) = \sum_{k=0}^{\infty} = \frac{(-1)^k}{(2k)!} z^{2k} = \frac{\exp(iz)+\exp(-iz)}{2}
	\]
	\[
	\sin(z) = \sum_{k=0}^{\infty} = \frac{(-1)^k}{(2k + 1)!} z^{2k + 1} = \frac{\exp(iz)-\exp(-iz)}{2i}
	\]
	sowie $\cos(-z) = \cos(z)$ und $\sin(-z) = -\sin(z)$.
\end{frameddefn}

\begin{framedthm}[Additionstheoreme]
	Sei $z_1,z_2 \in \C$.
	\[
	\cos(z_1 + z_2) = \cos(z_1)\cos(z_2) - \sin(z_1)\sin(z_2)
	\]
	\[
	\sin(z_1 + z_2) = \sin(z_1)\cos(z_2) + \sin(z_2)\cos(z_1)
	\]
\end{framedthm}

\begin{framedthm}[Analytische Eigenschaften von Sinus und Cosinus]
	\begin{enumerate}
		\item [(i)] $\sin$ und $\cos$ sind auf $\C$ stetig
		\item [(ii)] $\forall x \in \R\st \sin^2 x + \cos^2 x = 1$
		\item [(iii)] $r_{2n+2}$ und $r_{2n+3}$ bezeichnen die Restglieder von Cosinus und Sinus, also 
		\[
		\cos(z) = \sum_{k=0}^{n} = \frac{(-1)^k}{(2k)!} x^{2k} + r_{2n+2}(x)
		\]
		\[
		\sin(z) = \sum_{k=0}^{n} = \frac{(-1)^k}{(2k + 1)!} x^{2k + 1} + r_{2n+3}(x)
		\]
		Es gilt:
		\[
		\forall |x| < 2n+3\st |r_{2n+2}| \leq \frac{|x|^{2n+2}}{(2n+2)!}
		\]
		\[
		\forall |x| < 2n+4\st |r_{2n+3}| \leq \frac{|x|^{2n+3}}{(2n+3)!}
		\]
	\end{enumerate}
\end{framedthm}

\begin{framedthm}[Nullstelle des Cosinus]
	$\cos\st [0,2] \to \R$ hat genau eine Nullstelle, nämlich $\frac{\pi}{2}$.
\end{framedthm}

\begin{framedthm}[Folgerungen aus den Additionstheoremen]
	\begin{itemize}
		\item $\sin x_1 + \sin x_2 = 2 \sin(\frac{x_1 + x_2}{2}) \cos(\frac{x_1 - x_2}{2})$
		\item $\sin x_1 - \sin x_2 = 2 \cos(\frac{x_1 + x_2}{2}) \sin(\frac{x_1 - x_2}{2})$
		\item $\cos x_1 + \cos x_2 = 2 \cos(\frac{x_1 + x_2}{2}) \cos(\frac{x_1 - x_2}{2})$
		\item $\cos x_1 - \cos x_2 = -2 \sin(\frac{x_1 + x_2}{2}) \sin(\frac{x_1 - x_2}{2})$
	\end{itemize}
\end{framedthm}

\begin{framedquest}[Folgerungen der Nullstelle des Cosinus]
	\begin{itemize}
		\item $\exp(i\frac{\pi}{2}) = i$
		\item $\exp(i\pi) = -1$
		\item $\forall z \in \C\st \exp(z + 2\pi k i) = \exp(z)$ wenn $k \in \Z$. Gilt auch für $\sin$ und $\cos$
		\item $\sin(\frac{\pi}{2} - z) = \cos(z)$ sowie $\cos(\frac{\pi}{2} - z) = \sin(z)$
		\item $\cos(z) = 0 \ \iff\ z \in \{k\pi + \frac{\pi}{2} \ |\ k \in \Z\}$\\ $\sin(z) = 0 \ \iff\ z \in \{k\pi \ |\ k \in \Z\}$
	\end{itemize}
\end{framedquest}

\begin{framedthm}[Umkehrfunktion von Sinus und Cosinus]
	\begin{enumerate}
		\item[(i)] Die Funktion $\cos\st [0,\pi] \to [-1,1]$ ist streng monoton fallend und bildet das Intervall $[0,\pi]$ bijektiv auf $[-1,1]$ ab. Die Umkehrfunktion ist \\$\arccos\st [-1,1] \to [0,\pi]$
		\item[(ii)] Die Funktion $\sin\st [-\frac{\pi}{2},\frac{\pi}{2}] \to [-1,1]$ ist streng monoton wachsend und bildet das Intervall $[-\frac{\pi}{2},\frac{\pi}{2}]$ bijektiv auf $[-1,1]$ ab. Die Umkehrfunktion ist \\$\arcsin\st [-1,1] \to [-\frac{\pi}{2},\frac{\pi}{2}]]$
	\end{enumerate}
\end{framedthm}

\begin{framedquest}[Gültigkeitsbereich der Umkehrfunktion]
	Es gilt $\forall x \in [-1,1]\st \cos(\arccos(x)) = x$ und $\forall x \in [0,\pi]\st \arccos(\cos(x)) = x$.
\end{framedquest}

\begin{frameddefn}[Tangens]
	\begin{enumerate}
		\item [(i)] $\tan \st \C \setminus \{k\pi + \frac{\pi}{2} \ |\ k \in \Z\} \to \C$ definiert durch 
		\[
		\tan z = \frac{\sin z}{\cos z}
		\]
		$\tan\st ]-\frac{\pi}{2}, \frac{\pi}{2}[ \,\to \R$ ist streng monoton wachsend. Damit ist die Umkehrfunktion $\arctan\st \R \to\, ]-\frac{\pi}{2}, \frac{\pi}{2}[$.
		\item [(ii)] $\cot \st \C \setminus \{k\pi \ |\ k \in \Z\} \to \C$ definiert durch
		\[
		\cot z = \frac{\cos z}{\sin z}
		\]
	\end{enumerate}
\end{frameddefn}
	
	\chapter{Starre Körper}
	\begin{frameddefn}[Differenzierbarkeit einer Funktion]
	Sei $V \subset \R$ eine Menge und $f\st V \to \R$ oder $\C$ eine Funktion. Dann heißt $f$ in einem Punkt $x_0 \in V$ differenzierbar falls $x_0$ ein Häufungspunkt von $V$ ist und 
	\[
	f'(x_0) = \lim\limits_{x \to x_0} \frac{f(x) - f(x_0)}{x - x_0} \qquad (x \neq x_0)
	\] 
	existiert.
\end{frameddefn}

\section{Allgemeine Eigenschaften und wichtige Ableitungsregeln}

\begin{framedthm}[Stetigkeit von differenzierbaren Funktionen]
	Sei $f\st I \to \R$ oder $\C$ eine Funktion, die in $x_0 \in I$ differenzierbar ist. Dann ist $f$ in $x_0$ stetig.
\end{framedthm}

\begin{framedthm}[Lineare Approximation]
	Sei $f\st I \to \R$ eine Funktion. Dann ist $f$ genau dann in $x_0 \in I$ differenzierbar, wenn es eine Konstante $c \in \R$ gibt, so dass $f(x) = f(x_0) + c(x-x_0) + \textit{o}(|x-x_0|)$ für $x \to x_0$ (mit $c=f'(x_0)$).
\end{framedthm}


\begin{framedthm}[Algebraische Operationen]
	Seien $f_1, f_2 \st I \to \R$ Funktionen die in $x_0$ differenzierbar sind und $\lambda \in \R$ sowie\\ $f_2(x_0) \neq 0$. Dann sind folgende Operationen auch differenzierbar:
	\begin{enumerate}
		\item [(i)] $(f_1 + f_2)'(x_0) = f_1'(x_0) + f_2'(x_0)$
		\item [(ii)] $(\lambda f_1)'(x_0) = \lambda f_1 ' (x_0)$
		\item [(iii)] $(f_1 \cdot f_2)'(x_0) = f_1'(x_0) f_2(x_0) + f_1(x_0) f_2'(x_0)$
		\item [(iv)] $\left(\frac{f_1}{f_2}\right)'(x_0) = \frac{f_1'(x_0) f_2(x_0) - f_1(x_0) f_2'(x_0)}{f_2^2(x_0)}$
	\end{enumerate}
\end{framedthm}


\begin{framedthm}[Kettenregel]
	Sei $f_1\st I_1 \to \R$ und $f_2\st I_2 \to \R$, sodass $f_1(I_1) \subset I_2$. Die Funktion $f_1$ sei in $x_1 \in I_1$ differenzierbar und $f_2$ in $x_2 = f(x_1)$ differenzierbar. Dann ist $g(x) = (f_1 \circ f_2)(x)$ in $x_2$ differenzierbar und $g'(x) = f_2'(f_1(x_1)) f_1'(x_1)$.
\end{framedthm}

\begin{framedthm}[Ableitung der Umkehrfunktion]
	Sei $f\st I \to J$ bijektiv und stetig. Wenn $f$ im Punkt $x_0 \in I$ differenzierbar ist und $f'(x_0) \neq 0$, dann ist $f^{-1} \st J \to I$ im Punkt $y_0 = f(x_0) \in J$ differenzierbar und
	\[
	(f^{-1} (y_0))' = \frac{1}{f'(f^{-1}(y_0))} = \frac{1}{f'(x_0)}
	\]
\end{framedthm}

\begin{frameddefn}[Höhere Ableitungen und stetige Differenzierbarkeit]
	Sei $f\st I \to \R$ differenzierbar. Falls $f' \st I \to \R$ in $x_0$ differenzierbar ist, so heißt
	\[
	(f')'(x_0) = f''(x_0)
	\]
	die zweite Ableitung.\\
	Mit Induktion: $f\st I \to \R$ heißt $k$-mal differenzierbar in $x_0$, falls die $(k-1)$-te Ableitung $f^{(k-1)} \st I \to \R$ in $x_0$ differenzierbar ist, $f^{(k)}(x_0) = (f^{(k-1)}(x_0))'$.\\
	\begin{itemize}
		\item $f$ heißt $k$-mal differenzierbar, wenn $f$ in jedem Punkt aus $I$ $k$-mal differenzierbar ist.
		\item $f$ heißt $k$-mal \textit{stetig differenzierbar} in $I$, wenn $f$ $k$-mal differenzierbar ist und $f^{(k)}\st I \to \R$ stetig ist.
	\end{itemize}
\end{frameddefn}

\newpage
\section{Die zentralen Sätze der Differentialrechnung}

\begin{frameddefn}[Lokale Extrema]
	Sei $I \subset \R$ ein Intervall, $f\st I \to \R$ eine reelle Funktion. Man sagt, $f$ hat in $x_0 \in I$ ein
	\begin{enumerate}
		\item [(i)] lokales Maximum (bzw. streng lokales Maximum) wenn\\ $\exists \varepsilon_0 > 0\st \forall x \in I \cap ]x_0 - \varepsilon_0, x_0 + \varepsilon_0[\st f(x) \leq f(x_0)$ (bzw. $f(x) < f(x_0)$ mit $x \neq x_0$)
		\item [(ii)] lokales Minimum (bzw. streng lokales Minimum) wenn\\ $\exists \varepsilon_0 > 0\st \forall x \in I \cap ]x_0 - \varepsilon_0, x_0 + \varepsilon_0[\st f(x) \geq f(x_0)$ (bzw. $f(x) > f(x_0)$ mit $x \neq x_0$)
	\end{enumerate}
\end{frameddefn}

\begin{framedthm}[Satz von Fermat]
	Sei $f\st I \to \R$ differenzierbar in $x_0 \in I$ und sei in $x_0$ ein lokales Extrema sowie $\exists \delta_0 \st ]x_0 - \delta_0, x_0 + \delta_0[ \subset I$. Dann gilt $f'(x_0) = 0$.
\end{framedthm}

\begin{framedthm}[Satz von Rolle]
	Sei $f\st [a,b] \to \R$ eine stetige Funktion mit $f(a)=f(b)$. Wenn $f$ in $]a,b[$ differenzierbar ist, dann gilt: $\exists x_0 \in ]a,b[\st f'(x_0) = 0$.
\end{framedthm}

\begin{framedthm}[Mittelwertsatz der Differentialrechnung]
	Sei $f\st [a,b] \to \R$ eine stetige Funktion die auf dem Intervall $]a,b[$ differenzierbar ist. Dann gilt: \[
	\exists x_0 \in ]a,b[\st f'(x_0) = \frac{f(b) - f(a)}{b-a}
	\]
\end{framedthm}

\begin{framedthm}[Satz von Cauchy]
	Seien $f,g\st [a,b] \to \R$ stetige Funktionen die auf $[a,b]$ differenzierbar sind. Dann gilt:\\
	$\exists x_0 \in ]a,b[\st f'(x_0) (g(b)-g(a)) = g'(x_0) (f(b)-f(a))$
\end{framedthm}

\begin{framedthm}[Monotonie von Funktionen]
	Sei $f\st [a,b] \to \R$ eine stetige Funktion die auf $]a,b[$ differenzierbar ist. Dann gilt:
	\begin{enumerate}
		\item [(i)] $\forall x \in ]a,b[\st f'(x) > 0 \ \Rightarrow \ f$ ist \textit{streng} monoton wachsend. 
		\item [(ii)] $f$ ist monoton wachsend $\Rightarrow \ \forall x \in ]a,b[ \st f'(x) \geq 0$.
	\end{enumerate}
\end{framedthm}

\begin{framedthm}[Minimum bzw. Maximum einer Funktion]
	Sei $f\st]a,b[ \to \R$ differenzierbar. Wenn $x_0 \in ]a,b[$ existiert, so dass $f'(x_0) = 0$ und $f''(x_0)$ existiert mit $f''(x_0) > 0$ (bzw. $f''(x_0) < 0$. Dann hat $f$ in $x_0$ ein streng lokales Minimum (bzw. Maximum).
\end{framedthm}

\section{Konvexität}

\begin{frameddefn}[Konverxe und konkave Funktionen]
	Sei $I \subset \R$. $f\st I \to \R$ heißt \textit{konvex}, falls $\forall x_0, x_1 \in I\st \forall \lambda \in [0,1]\st f(\lambda x_1 + (1-\lambda) x_0) \leq \lambda f(x_1) + (1-\lambda) f(x_0)$. $f$ heißt \textit{konkav}, wenn $-f$ konvex ist.
\end{frameddefn}

\begin{framedthm}[Kriterium für Konvexität]
	Sei $f\st I \to \R$, wobie $I$ ein offenes Intervall ist und $f$ zwei mal differenzierbar ist:
	\[
	f \textrm{ ist konvex } \iff\  \forall x \in I\st f''(x) \geq 0 \ \iff\ f' \textrm{ ist monoton wachsend}
	\]
\end{framedthm}

\begin{framedthm}[Regel von l'Hôspital]
	Seien $f,g \st ]a,b[ \to \R$ differenzierbare Funktionen und $\forall x \in ]a,b[\st g'(x) \neq 0$ (dabei ist $a=-\infty$ und $b=\infty$ zugelassen). Gilt dann:
	\[
	\lim\limits_{x \to a} f(x) = \lim\limits_{x \to a} g(x) = 0 \textrm{ oder } \lim\limits_{x \to a} g(x) = \infty
	\]
	und existiert der Grenzwert:
	\[
	\lim\limits_{x \to a} \frac{f'(x)}{g'(x)} = L \textrm{ wobei } x > a \textrm{ und } -\infty\leq L \leq \infty
	\]
	Dann gilt:
	\[
	\lim\limits_{x \to a} \frac{f(x)}{g(x)} = L \textrm{ wobei } x > a
	\]
	\textbf{Hinweis:} Es müssen \textit{alle} Voraussetzungen erfüllt sein (d.h. differenzierbar, $g'(x) = 0$, Limes existiert, Zähler und Nenner gehen gegen $0$ oder $\infty$). Die Regel gilt ebenso wenn $x \to b$ mit $x < b$.
\end{framedthm}

\newpage
\subsection{Taylor Reihe}

\begin{framedthm}[Satz von Taylor]
	Sei $f\st I \to \R$ eine Funktion und $x_0, x \in I$. Sei $I_0 = [x, x_0]$ falls $x < x_0$ ($[x_0, x]$ falls $x_0 < x$) und $J_0$ das offene Intervall von $I_0$. Die Funktion $f_{|I_{0}}$ und ihre ersten $n$ Ableitungen seien auf $I_0$ stetig und $f^{(n)}_{|J_{0}}$ ist differenzierbar. Dann existiert ein $\xi$ zwischen $x$ und $x_0$, sodass gilt:
	\[
	f(x) = \sum^{n}_{k=0} \frac{f^{(n)}(x_0)}{k!} (x-x_0)^k + r_n(x_0, x)
	\]
	wobei
	\[
	r_n(x_0,x) = \frac{f^{(n+1)}(\xi)}{(n+1)!} (x-x_0)^{n+1} \textrm{ (Restglied nach Lagrange)}
	\]
	oder
	\[
	r_n(x_0,x) = \frac{f^{(n+1)}(\xi)}{n!} (x-\xi)^{n} (x-x_0) \textrm{ (Restglied nach Cauchy)}
	\]
\end{framedthm}


	
	\chapter{Kleine Schwingungen}
	\section{Allgemeine Eigenschaften integrierbarer\\ Funktionen}

\begin{frameddefn}[Treppenfunktion]
	Sei $a < b$, $a,b \in \R$ und $\phi\st [a,b] \to \R$. $\phi$ heißt Treppenfunktion, wenn es eine Unterteilung des Intervalls $[a,b]$ mit $a= x_0 < x_1 < x_2 < ... < x_n = b$ gibt und $c_1,...,c_n \in \R$ existieren, sodass $\phi_{|]x_{k-1}, x_{k}[} = c_k$ mit $k= 1,..,n$.\\
	$T[a,b]$ ist der Vektorraum der Treppenfunktionen.
\end{frameddefn}

\begin{frameddefn}
	Sei $\phi \in T[a,b]$ und $a=x_0 < x_1 < ... < x_k = b$ und $c_i = \phi_{|]x_{i-1}, x_{i}[}$. Dann:
	\[
	\int_{a}^{b} \phi(x) \, dx := \sum_{i=1}^{n} c_i (x_i - x_{i-1})
	\]
\end{frameddefn}

\begin{framedthm}
	Sei $\phi, \psi \in T[a,b]$ und $\lambda \in \R$. Dann gilt:
	\begin{itemize}
		\item $\int_{a}^{b} (\phi(x) + \psi(x))dx = \int_{a}^{b} \phi(x) dx + \int_{a}^{b} \psi(x) dx$
		\item $\int_{a}^{b} (\lambda\phi(x))dx = \lambda \int_{a}^{b} \phi(x)dx$
		\item $\phi \geq \psi \ \Rightarrow\ \int_{a}^{b} \phi(x)dx \geq \int_{a}^{b} \psi(x)dx$
	\end{itemize}
\end{framedthm}

\begin{frameddefn}[Ober- und Unterintegral]
	Sei $f\st [a,b] \to \R$ eine beliebige, aber beschränkte Funktion. Dann:
	\begin{itemize}
		\item $\overline{\int_{a}^{b}} f(x)dx = \inf\{\int_{a}^{b} \phi(x)dx \ |\ \phi \in T[a,b], \phi \geq f\}$ (Oberintegral)
		\item $\underline{\int_{a}^{b}} f(x)dx = \sup\{\int_{a}^{b} \psi(x)dx \ |\ \psi \in T[a,b], \psi \leq f\}$ (Unterintegral)
	\end{itemize}
	$f$ heißt Riemann-integrierbar, wenn:
	\[
	\overline{\int_{a}^{b}} f(x)dx = \underline{\int_{a}^{b}} f(x)dx =: \int_{a}^{b} f(x)dx
	\]
\end{frameddefn}

\begin{framedthm}
	$f\st [a,b] \to \R$ ist (Riemann) integrierbar $\iff$ $\forall \varepsilon > 0\st \exists \phi,\psi \in T[a,b]\st \psi \leq f \leq \phi$ und\\
	$\int_{a}^{b} \phi(x)dx - \int_{a}^{b} \psi(x)dx < \varepsilon$
\end{framedthm}

\begin{framedthm}
	Jede stetige Funktion ist integrierbar.
\end{framedthm}

\begin{framedthm}
	Jede monotone Funktion ist integrierbar.
\end{framedthm}

\begin{framedthm}
	Seien $f,g\st [a,b] \to \R$ zwei integrierbare Funktionen und $\lambda \in \R$. Dann sind auch folgende Funktionen integrierbar:
	\begin{enumerate}
		\item [(i)] $f+g$
		\item [(ii)] $\lambda f$
		\item [(iii)] $f_+$, $f_{-}$
		\item [(iv)] $\forall p \geq 1\st |f|^p$
	\end{enumerate}
	Außerdem gilt:
	\begin{itemize}
		\item $f \geq g \ \Rightarrow \ \int_{a}^{b} f(x) dx \geq \int_{a}^{b} g(x) dx$
		\item $|\int_{a}^{b} f(x) dx| \leq \int_{a}^{b} |f(x)| dx $
	\end{itemize}
\end{framedthm}

\begin{framedthm}[Mittelwertsatz der Integralrechnung]
	Sei $f\st[a,b] \to \R$ stetig, dann  $\exists \xi \in [a,b]\st \int_{a}^{b} f(x)dx = (b-a)f(\xi)$.
\end{framedthm}

\begin{frameddefn}[Riemannsche Summen]
	Sei $[a,b] \subset \R$ und $a=x_0 < x_1 < ... < x_n = b$ eine Unterteilung des Intervalls $[a,b]$. Sei außerdem $\xi_k \in [x_{k-1}, x_k]$, $(\xi_k)_{k=1,...,n}$ eine Stützstelle. Dann definiert man die Riemannsche Summe der Funktion $f$ zur Unterteilung $(x_i)_{i=0,...,n}$ und Stützstelle $(\xi_i)_{i=1,...,n}$ folgendermaßen:
	\[
	\mathcal{R}_f((x_k), (\xi_k)) := \sum_{k=1}^n f(\xi_k) (x_k - x_{k-1}) 
	\]
	$\mu((x_i)) = \max(x_i - x_{i-1})_{i=1,...,n}$ gibt die Feinheit der Unterteilung an.
\end{frameddefn}

\begin{framedthm}
	Sei $f\st[a,b] \to \R$ eine integrierbare Funktion.
	\[
	\forall \varepsilon > 0\st \exists \delta > 0\st \forall (x_i)_{i=0,...,n}\st \forall (\xi_i)_{i=1,...,n}\st \mu((x_i)) < \delta \Rightarrow \biggl|\int_a^b f(x) dx - \mathcal{R}_f((x_i), (\xi_i))\biggr| < \varepsilon
	\]
	Dies erlaubt es, dass Riemannsche Integral als Grenzwert zu betrachten.
\end{framedthm}


\section{Zusammenhang zwischen Integral und Ableitung}

\begin{framedthm}
	Sei $f\st[a,b] \to \R$ eine integrierbare Funktion. Dann:
	\[
	\forall x_1,x_2 \in [a,b]\st \int_{x_{1}}^{x_{2}} f(x) dx = - \int_{x_{2}}^{x_{1}} f(x)dx
	\]
	\[
	\forall x_1, x_2, x_3 \in [a,b]\st \int_{x_{1}}^{x_{3}} f(x) dx = \int_{x_{1}}^{x_{2}} f(x) dx + \int_{x_{2}}^{x_{3}} f(x) dx
	\]
	\[
	F(x) := \int_a^x f(t) dt
	\]
\end{framedthm}

\begin{framedthm}
	Sei $f\st[a,b] \to \R$ eine integrierbare Funktion und sei $f$ in $x_0 \in [a,b]$ stetig. Dann ist die Funktion $F(x) = \int_a^x f(t) dt$ in $x_0$ differenzierbar und $F'(x_0) = f(x_0)$.
\end{framedthm}

\begin{frameddefn}[Stammfunktion]
	Eine differenzierbare Funktion $F\st[a,b] \to \R$ heißt Stammfuntkion oder primitive Funktion einer Funktion $f\st[a,b] \to \R$ falls $\forall x\in [a,b]\st F'(x) = f(x)$ gilt.
\end{frameddefn}

\begin{framedthm}[Fundamentalsatz der Differential- und Integralrechnung]
	Sei $f\st[a,b] \to \R$ eine stetige Funktion und sei $F\st[a,b] \to \R$ eine Stammfunktion von $f$. Dann gilt:
	\[
	\int_a^b f(x) dx = F(b) - F(a) =: (F(x)) \bigr|_a^b
	\]
\end{framedthm}

\begin{framedthm}[Integration durch Substitution]
	Sei $f\st I \to \R$ stetig und $\phi\st [a,b] \to \R$ stetig differenzierbar sowie $\phi([a,b]) = I$. Dann:
	\[
	\int_a^b f(\phi(x)) \phi'(x) dx = \int_{\phi(a)}^{\phi(b)} f(t) dt
	\]
\end{framedthm}

\begin{framedthm}[Partielle Integration]
	Seien $f,g\st [a,b] \to \R$ stetig differenzierbare Funktionen. Dann:
	\[
	\int_a^b f'(x)g(x) dx = (f(x)g(x))\bigr|_a^b - \int_a^b f(x) g'(x) dx
	\]
\end{framedthm}
	
	\chapter{Hamiltonformalismus}
	
\begin{frameddefn}[Hamilton-Funktion]
	\[H(q,p,t) = \sum^f_{i=1} \dot q_i(q,p,t) \, p_i - \mathscr{L}(q, \dot q(q,p,t), t) \]
	mit dem verallgemeinerten Impuls $p_i = \frac{\partial \mathscr{L}}{\partial \dot q_i}$
\end{frameddefn}

\begin{framedprop}[Hamilton-Gleichungen]
	\[ \dot p_k = - \frac{\partial H}{\partial q_k} \]
	\[ \dot q_k = \frac{\partial H}{\partial p_k} \]
	
	Im Gegensatz zu den $f$ DGLs 2. Ordnung im Lagrangeformalismus erhält im Hamiltonformalismus $2f$ DGLs 1. Ordnung.
\end{framedprop}

\begin{framedthm}[Energieerhaltung]
Falls in der kinetischen Energie die generalisierte Geschwindigkeit nur quadratisch vorkommt und die potenzielle Energie nur von den generalisierten Koordinaten abhängt, gilt:

\[ H(q,p,t) = T + U = E \]

\[ \frac{\dd H}{\dd t} = \frac{\partial H}{\partial t} = -\frac{\partial \mathscr{L}}{\partial t} \ \textrm{ und damit }\  \frac{\partial \mathscr{L}}{\partial t} = 0 \ \Rightarrow \ H = \textrm{const}\]
	
\end{framedthm}

	
	\chapter{Spezielle Relativitätstheorie}
	\begin{framedthm}[Einsteins Relativitätsprinzip]

	\begin{itemize}
		\item Alle Inertialsysteme sind gleichwertig
		\item Licht hat in jedem Inertialsystem die gleiche Geschwindigkeit $c = 2,998 \cdot 10^8 \frac{\textrm{m}}{\textrm{s}}$
	\end{itemize}
	
\end{framedthm}

\begin{frameddefn}[Abstand zwischen zwei Ereignissen]
	
	\[ s_{12}^2 = c^2(t_2 - t_1)^2 - (x_2 - x_1)^2 - (y_2 - y_1)^2 - (z_2 - z_1)^2 \]
	
	Der raum-zeitliche Abstand zwischen zwei festen Ereignissen 	ist gleich, egal in welchem Inertialsystem.
	
\end{frameddefn}

\begin{frameddefn}[Lorentztransformation]
	
	\[ ct' = \gamma \left(ct - \frac{xv}{c}\right) \qquad x' = \gamma (x - vt) \qquad y' = y \qquad z' = z \]
	
	mit $\gamma = \frac{1}{\sqrt{1 - v^2/c^2}}$ \\
	
	Daraus folgt die \textbf{Längenkontraktion} (Komponente parallel zur Bewegungsrichtung erscheint verkürzt):
	
	\[ l = l_0 \sqrt{1-v^2/c^2} \]
	
	Und die \textbf{Zeitdilatation} (bewegte Uhr geht langsamer):
	
	\[ t = \frac{t_0}{\sqrt{1 - v^2/c^2}} \]
	
	Der Index 0 bezeichnet dabei jeweils die Größe im ruhenden Inertialsystem.
	
\end{frameddefn}


\end{document}