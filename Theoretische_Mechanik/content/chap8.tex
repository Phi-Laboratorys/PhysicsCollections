\begin{framedthm}[Einsteins Relativitätsprinzip]

	\begin{itemize}
		\item Alle Inertialsysteme sind gleichwertig
		\item Licht hat in jedem Inertialsystem die gleiche Geschwindigkeit $c = 2,998 \cdot 10^8 \frac{\textrm{m}}{\textrm{s}}$
	\end{itemize}
	
\end{framedthm}

\begin{frameddefn}[Abstand zwischen zwei Ereignissen]
	
	\[ s_{12}^2 = c^2(t_2 - t_1)^2 - (x_2 - x_1)^2 - (y_2 - y_1)^2 - (z_2 - z_1)^2 \]
	
	Der raum-zeitliche Abstand zwischen zwei festen Ereignissen 	ist gleich, egal in welchem Inertialsystem.
	
\end{frameddefn}

\begin{frameddefn}[Lorentztransformation]
	
	\[ ct' = \gamma \left(ct - \frac{xv}{c}\right) \qquad x' = \gamma (x - vt) \qquad y' = y \qquad z' = z \]
	
	mit $\gamma = \frac{1}{\sqrt{1 - v^2/c^2}}$ \\
	
	Daraus folgt die \textbf{Längenkontraktion} (Komponente parallel zur Bewegungsrichtung erscheint verkürzt):
	
	\[ l = l_0 \sqrt{1-v^2/c^2} \]
	
	Und die \textbf{Zeitdilatation} (bewegte Uhr geht langsamer):
	
	\[ t = \frac{t_0}{\sqrt{1 - v^2/c^2}} \]
	
	Der Index 0 bezeichnet dabei jeweils die Größe im ruhenden Inertialsystem.
	
\end{frameddefn}