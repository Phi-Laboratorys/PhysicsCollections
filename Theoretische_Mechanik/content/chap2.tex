\section{Lagrange Gleichungen 1. Art}

\begin{frameddefn}[Zwangsbedingungen]

Eine Zwangsbedingung schränkt die Koordinaten auf eine Bahn oder Ebene ein. Eine \textit{holonome} Zwangsbedingung hat folgende Form:

\[ g(\mathbf{r}, t) = 0 \]

Falls die Zwangsbedingung unabhängig von $t$ ist, nennt man sie \textit{skleronom}, ansonsten \textit{rheonom}.
	
\end{frameddefn}

\begin{frameddefn}[Lagrange Gleichungen 1. Art]
	
	Eine Zwangsbedingung wird durch eine Zwangskraft $\mathbf{Z} = \lambda(t) \nabla g(\mathbf{r}, t)$ sichergestellt. Für $R$ Zwangsbedingungen und $N$ Teilchen ergeben sich die Lagrange Gleichungen 1. Art zu:
	
	\[ m_n \ddot x_n = F_n + \sum^R_{\alpha=1} \lambda_\alpha(t) \frac{\partial g_\alpha(x_1,...,x_{3N}, t)}{\partial x_n}  \]
	
	Für ein Teilchen und nur eine Zwangsbedingung also:
	
	\[ m \mathbf{\ddot r} = \mathbf{F} + \mathbf{Z} = \mathbf{F} + \lambda(t) \nabla g(\mathbf{r}, t) \]
	
	Impuls und Drehimpuls sind erhalten wenn $\mathbf{F} + \mathbf{Z} = 0$ gilt. Außerdem gilt für die Energie:
	
	\[ \frac{\dd E}{\dd t} = - \sum_\alpha \lambda_\alpha \frac{\partial g_\alpha}{\partial t} \ \Rightarrow \ E = \textrm{const} \ \textrm{ wenn } \ \forall \alpha \st \frac{\partial g_\alpha}{\partial t} = 0 \]
	
\end{frameddefn}

\begin{framedmeth}[Allgemeins Vorgehen um 1. Art zu lösen]
	
	\begin{enumerate}
		\item Formulierung der Zwangsbedingungen durch $g_\alpha = 0$
		\item Aufstellen der Lagrange Gleichungen 1. Art
		\item Elimination der $\lambda_\alpha$ indem $\frac{\dd^2 g_\alpha}{\dd t^2}$ berechnet wird
		\item Lösung der Bewegungsgleichungen
		\item Bestimmung der Integrationskonstanten
		\item Bestimmung der Zwangskräfte mit $\mathbf{Z} = \lambda(t) \nabla g(\mathbf{r}, t)$
	\end{enumerate}
	
\end{framedmeth}


\section{Lagrange Gleichungen 2. Art}

\begin{frameddefn}[Verallgemeinerte Koordinaten]

Mit $R$ Zwangsbedingungen hat ein System noch $f = 3N - R$ Freiheitsgrade. Diese werden mit verallgemeinerten Koordinaten $q_1, ... , q_f$ dargestellt. Dazu ist eine Transformation 

\[ x_n = x_n(q_1, ..., q_f, t) = x_n(q,t) \]

notwendig. Die Zwangsbedingungen sowie die kinetische Energie $T = \sum_n \frac{m}{2} \dot x_n^2$ und die potenzielle Energie können dann mit verallgemeinerten Koordinaten formuliert werden:

\[ T(q, \dot q, t) \qquad U(q,t) \qquad g_\alpha(q,t) = 0 \]
	
\end{frameddefn}

\begin{frameddefn}[Lagrange Gleichungen 2. Art]

Die Lagrange Funktion wird wie folgt definiert:

\[ \mathscr{L}(q,\dot q, t) = T(q, \dot q, t) - U(q,t) \]

Eine Koordinate $q_k$ heißt \textit{zyklisch}, falls $\frac{\partial \mathscr{L}}{\partial q_k} = 0$.

Der verallgemeinerte (oder auch kanonische) Impuls ist $p_k = \frac{\partial \mathscr{L}}{\partial \dot q_k}$.

Die Lagrange Gleichungen 2. Art sind dann $f$ DGLs 2. Ordnung:

\[ \frac{\dd}{\dd t} \frac{\partial \mathscr{L}}{\partial \dot q_k} = \frac{\partial \mathscr{L}}{\partial q_k} \]
	
\end{frameddefn}

\newpage
\section{Hamiltonsches Prinzip}

\begin{frameddefn}[Wirkungsfunktional]
	Das Wirkungsfunktional ist wie folgt definiert:
	
	\[ S[q] = \int_{t_1}^{t_2} \dd t \, \mathscr{L}(q, \dot q, t) \]
	
	Das Prinzip der kleinsten Wirkung ($\delta S[q] = 0$) führt als Variationsproblem wieder zu den Lagrange Gleichungen.
\end{frameddefn}


\section{Eichtransformation}

\begin{framedthm}[Eichtransformation]
	
	Verschiedene Lagrange Funktionen führen zu denselben Bewegungsgleichungen:
	
	\[ \delta \int_{t_1}^{t_2} \dd t \, \mathscr{L} = \delta \int_{t_1}^{t_2} \dd t \, \mathscr{L}^* = 0 \]
	
	wenn z.B. $\mathscr{L}^* = \textrm{const} \cdot \mathscr{L}$ oder $\mathscr{L}^* = \mathscr{L} + \textrm{ const}$. Allgemein führen alle Lagrange Funktionen auf dieselben Bewegungsgleichungen bei folgender Form:
	
	\[ \mathscr{L}^*(q, \dot q, t) = \mathscr{L}(q, \dot q, t) + \frac{\dd}{\dd t} f(q,t) \]
	
\end{framedthm}

\section{Noether Theorem}

\begin{framedthm}[Erhaltungsgröße des Noether Theorems]
	
	Das Noether Theorem besagt, wie aus Symmetrien Erhaltungsgrößen folgen. Betrachte dazu folgende Transformation:
	
	\[ q_i^* = q_i + \varepsilon \psi_i(q,\dot q, t) + \mathcal{O}(\varepsilon^2) \]
	
	\[ t^* =  t + \varepsilon \varphi(q, \dot q, t) + \mathcal{O}(\varepsilon^2) \]
	
	Bei der starken Annahme, dass die zwei Wirkungsfunktionale gleich sind ($S^*[q^*(t^*)] = S[q(t)]$) ergibt sich die folgende Erhaltungsgröße:
	
	\[ Q(q,\dot q, t) = \sum_{i=1}^f \frac{\partial \mathscr{L}}{\partial \dot q_i} \psi_i + (\mathscr{L} - \sum^f_{i=1} \frac{\partial \mathscr{L}}{\partial \dot q_i} \dot q_i) \, \varphi = \textrm{const} \]
	
\end{framedthm}