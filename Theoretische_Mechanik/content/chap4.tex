\section{Herleitung Bewegungsgleichungen und Lösung}

\begin{framedprop}[Herleitung der Bewegungsgleichungen]
	
	Mit der Schwerpunkt und Relativkoordinate lässt sich die Lagrangefunktion entkoppeln:
	
	\[ \mathbf{R} = \frac{m_1 \mathbf{r}_1 + m_2 \mathbf{r}_2}{m_1+m_2}, \quad \mathbf{r} = \mathbf{r}_1 - \mathbf{r}_2 \ \Rightarrow \ \mathscr{L} = \mathscr{L}_1(\mathbf{R}, \mathbf{\dot R}) + \mathscr{L}_2(\mathbf{r}, \mathbf{\dot r})\]
	
	$\mathbf{R}$ ist zyklisch, damit ist der Schwerpunktimpuls erhalten und es folgt: $\mathbf{R}(t) = \mathbf{A} t + \mathbf{B}$. Die Lagrange Gleichungen liefern die Bewegungsgleichungen für die Relativbewegung. Durch die Rotations- und Zeitsymmetrie folgt aber Drehimpuls- und Energieerhaltung, wodurch sich insgesamt eine DGL 1. Ordnung ergibt (jeweils für $\varphi$ und $\rho$ wenn $\mathbf{r}$ in Zylinderkoordinaten dargestellt wird).
	
	Drehimpulserhaltung:
	
	\[ l = \mu \rho^2 \dot\varphi \qquad \Rightarrow \varphi(t)\]
	
	Energieerhaltung:
	
	\[ E = \frac{\mu}{2} \dot\rho^2 + \frac{l}{2\mu\rho^2} + U(\rho) \]
	
	Mit $\frac{\dd \varphi}{\dd \rho} = \frac{\partial \varphi}{\partial t}\frac{\dd t}{\dd \rho}$ folgt:
	
	\[ \varphi(\rho) = \varphi_0 + \int_{\rho_0}^{\rho} \dd \rho' \frac{l/\rho'^2}{\sqrt{2\mu (E-U) - l^2/\rho'^2}} \]
	
\end{framedprop}

\newpage
\section{Verschiedene Zentralpotenziale}

\begin{frameddefn}[Wichtige Potenziale]
	
	\begin{enumerate}
		\item Harmonischer Oszillator: $U(\rho) = \alpha (\rho-\rho_0)^2 \ \Rightarrow$ gebundene Bewegung
		\item $U(\rho) = - \alpha / \rho$ (z.B. Gravitations- oder elektrostatisches Potenzial)\\
		Falls $\alpha < 0$ ist das Potenzial rein repulsiv, die Bewegung also ungebunden. \\
		Falls $\alpha > 0$ ist die Bewegung gebunden wenn $E < 0$, bei $E > 0$ ist die Bewegung ungebunden.
	\end{enumerate}
	
\end{frameddefn}

\section{Keplerproblem}

\begin{frameddefn}[Parameter einer Ellipse]
	
	Exzentrizität $\varepsilon$ und Paramter $p$ ergeben die Ellipsengleichung
	
	\[ p = \frac{l^2}{\mu \alpha} \qquad \varepsilon = \sqrt(1 + \frac{2 E l^2}{\mu \alpha^2}) \qquad \Rightarrow \qquad \frac{p}{\rho} = 1 + \varepsilon \cos \varphi \]
	
	Es gilt:
	\begin{itemize}
		\item $\varepsilon = 0 \ \Rightarrow$ Kreis
		\item $\varepsilon < 1 \ \Rightarrow$ Ellipse
		\item $\varepsilon = 1 \ \Rightarrow$ Parabel
		\item $\varepsilon > 1 \ \Rightarrow$ Hyperbel
	\end{itemize}
	
	Große Halbachse $a$, kleine Halbachse $b$:
	\[ a = \frac{p}{1 - \varepsilon^2} \qquad b = \frac{p}{\sqrt{1 - \varepsilon^2}} \qquad \Rightarrow \qquad \frac{(x+a\varepsilon)^2}{a^2} + \frac{y^2}{b^2} = 1\]
	
\end{frameddefn}

\begin{framedthm}[Keplersche Gesetze]
	\begin{enumerate}
		\item Die Planetenbahnen sind Ellipsen mit der Sonne als Brennpunkt
		\item Ein Fahrstrahl überstreicht bei gleicher Zeit die gleiche Fläche
		\item Die Quadrate der Umlaufdauern sind proportional zu den Kuben der großen Halbachse
	\end{enumerate}
\end{framedthm}


