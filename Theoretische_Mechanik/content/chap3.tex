 \section{Euler Lagrange Gleichung}
 
 \begin{frameddefn}[Funktional]
 	Ein Funktional weißt jeder Funktion $y$ einen Wert zu und hat meist die Form:
 	
 	\[ J[y] = \int_{x_1}^{x_2} \dd x \, F(x, y, y') \]
 	
 	Meistens integriert man über eine Zeit $\dd t$ oder eine Länge $\dd s$ und nutzt dann $v = \frac{\dd s}{\dd t}$ und $\dd s^2 = \dd x^2 + \dd y^2 \ \Rightarrow \ \dd s = \dd x \sqrt{1+y'^2}$
 \end{frameddefn}
 
 \begin{framedthm}[Euler Lagrange Gleichungen]
 	
 	Um das $y(x)$ zu finden, bei dem $J$ extremal wird, stellt man die Euler Lagrange Gleichungen auf:
 	
 	\[ \frac{\dd}{\dd x} \frac{\partial F}{\partial y'} = \frac{\partial F}{\partial y} \]
 	
 	Falls $\frac{\partial F}{\partial y} = 0$ vereinfacht sich die Gleichung zu:
 	
 	\[ \frac{\partial F}{\partial y'} = \textrm{const} \]
 	
 	Falls $\frac{\partial F}{\partial x} = 0$ vereinfacht sich die Gleichung zu (Beltrami Identität):
 	
 	\[ y' \, \frac{\partial F}{\partial y'} - F = \textrm{const} \]
 	
 \end{framedthm}

\section{Variation mit Nebenbedingungen}

\begin{framedthm}[Isoperimetrische Nebenbedingungen]
	Eine isoperimetrische Nebenbedinung hat die Form
	
	\[ K[y] = \int_{x_1}^{x_2} \dd x \, G(x, y, y') \]
	
	Bei $R$ Nebenbedingungen ergibt sich die neue Funktion
	
	\[ F^*(x, y, y') = F - \sum_{i=1}^R \lambda_i G_i \]
	
	$y(x)$ lässt sich mit $F^*$ und der Euler-Lagrange Gleichung bestimmen.
\end{framedthm}

\begin{framedthm}[Holonome Nebenbedingungen]
	
	Eine holonome Nebenbedingung hat die Form
	
	\[ g(y, x) = 0 \]
	
	Diesmal gilt für $R$ Nebenbedingungen:
	
	\[ F^*(x, y, y') = F - \sum_{i=1}^R \lambda_i(x) g_i \]
	
\end{framedthm}


\section{Variation mit mehreren Variablen}


\begin{framedprop}[Mehrere unabhängige und abhängige Variablen]

\[ I[y_1, ..., y_n] = \int \dd x_1 ... \dd x_m \, F(x_1,...,x_m, y_1,...,y_n, Y_{11},..., Y_{nm}) \]

\[ \textrm{mit: } y_i = y_i(x_1,..., x_m) \qquad Y_{ij} = \frac{\partial y_i}{\partial x_j} \qquad i = 1,...,n\]

Damit ergeben sich dann die Euler-Lagrange Gleichungen:

\[ \sum_{j=1}^m \frac{\partial}{\partial x_j} \left(\frac{\partial F}{\partial Y_{ij}} \right) = \frac{\partial F}{\partial y_i} \qquad i = 1,...,n \]
	
\end{framedprop}




