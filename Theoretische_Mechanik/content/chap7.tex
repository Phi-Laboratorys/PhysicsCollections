
\begin{frameddefn}[Hamilton-Funktion]
	\[H(q,p,t) = \sum^f_{i=1} \dot q_i(q,p,t) \, p_i - \mathscr{L}(q, \dot q(q,p,t), t) \]
	mit dem verallgemeinerten Impuls $p_i = \frac{\partial \mathscr{L}}{\partial \dot q_i}$
\end{frameddefn}

\begin{framedprop}[Hamilton-Gleichungen]
	\[ \dot p_k = - \frac{\partial H}{\partial q_k} \]
	\[ \dot q_k = \frac{\partial H}{\partial p_k} \]
	
	Im Gegensatz zu den $f$ DGLs 2. Ordnung im Lagrangeformalismus erhält im Hamiltonformalismus $2f$ DGLs 1. Ordnung.
\end{framedprop}

\begin{framedthm}[Energieerhaltung]
Falls in der kinetischen Energie die generalisierte Geschwindigkeit nur quadratisch vorkommt und die potenzielle Energie nur von den generalisierten Koordinaten abhängt, gilt:

\[ H(q,p,t) = T + U = E \]

\[ \frac{\dd H}{\dd t} = \frac{\partial H}{\partial t} = -\frac{\partial \mathscr{L}}{\partial t} \ \textrm{ und damit }\  \frac{\partial \mathscr{L}}{\partial t} = 0 \ \Rightarrow \ H = \textrm{const}\]
	
\end{framedthm}
